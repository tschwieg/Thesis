\documentclass[12pt]{paper}



\usepackage[margin=1in]{geometry}
\usepackage{tikz}
\usepackage{natbib}
\usepackage{amsmath}
\usepackage{bm}
\usepackage{amsthm}
\usepackage{mathtools}
\usepackage{amsfonts}
\usepackage{bbm}
\usepackage{graphicx}

\DeclareMathOperator{\diam}{diam}
\DeclareMathOperator{\interior}{int}
\DeclareMathOperator{\close}{cl}

\newcommand{\met}[1]{d \left ( #1 \right )}
\newcommand{\brak}[1]{ \left [ #1 \right ] }
\newcommand{\cbrak}[1]{ \left \{ #1 \right \}}
\renewcommand{\vec}[1]{ \bm{ #1 }}
\newcommand{\abs}[1]{\left \lvert #1 \right \rvert}
\newcommand{\seq}[1]{{\left \{ #1 \right \}}}
\newcommand{\conj}[1]{ \overline{ #1 } }
%\newcommand{\close}[1]{ \bar{ #1 } }
\newcommand{\set}[1]{\left \{ #1 \right \}}
\newcommand{\Lim}{\lim\limits}
\newcommand{\compose}{\circ}
\newcommand{\inv}[1]{{#1}^{-1}}
\newcommand{\compl}[1]{{#1}^{c}}



\newcommand{\setR}{ \mathbb{R} }
\newcommand{\setQ}{ \mathbb{Q} }
\newcommand{\setZ}{ \mathbb{Z} }
\newcommand{\setN}{ \mathbb{N} }

\newcommand{\plim}{ \overset{p}{\to} }
\newcommand{\mean}[2][N]{ \overline{ #2 }_{#1}}
\newcommand{\exV}[1]{\mathbb{E} \left [ #1 \right ]}
\newcommand{\Vari}[1]{\mathbb{V} \left ( #1 \right )}

\newcommand{\est}[2][n]{ \widehat{ #2 }_{#1}}
\newcommand{\altest}[2][n]{ \tilde{ #2 }_{#1}}

\newcommand{\indicate}[1]{ \mathbbm{1}_{\{#1\}}}
\newcommand{\convDist}{ \overset{d}{\to}}
\newcommand{\unif}{\emph{U}}
\newcommand{\normal}{\mathcal{N}}
\newcommand{\eye}{\mathbbm{I}}

\newcommand{\bigO}{\mathcal{O}}
\newcommand{\Lagrange}{\mathcal{L}}

\newcommand{\deriv}[2]{\frac{ \partial #1}{ \partial #2}}

\DeclarePairedDelimiter{\ceil}{\lceil}{\rceil}
\DeclarePairedDelimiter{\floor}{\lfloor}{\rfloor}
\DeclarePairedDelimiter{\norm}{\lVert}{\rVert}

\newtheorem{assume}{Assumption}

\bibliographystyle{chicago}

\title{The benefits of Randomization Mechanisms in Counter-Strike:
  Global Offensive}
\author{Timothy Schwieg}

\begin{document}

\maketitle

\section{Research Question}


In the world of video games, a market has appeared for in-game
purchases. These cosmetic items affect the aesthetics of a player, but
often do not influence the game-play and are sold by the
designer. Recently the method of the sales has moved away from the
traditional market approach of individual prices for each item, and
towards the ``loot box'' approach. These items are sold in randomized
lotteries, often given away, with a cost of opening.

Traditional economic research on randomization informs us that for the
risk-neutral customer, there is no benefit to randomization, as the
consumer is indifferent. So for this mechanism to be so far-reaching
into the market, there must be a risk-loving nature to the
consumers. This begs the question of how much money are these
companies gaining by exploiting the risk-loving nature of the
consumers.

Counter-Strike global offensive presents an interesting case study for
these types of markets, as there is a secondary market where
individuals can buy and sell these loot boxes, as well as their
contents. This was one of the first games to introduce the concept of
the randomized ``loot box'' so there is a long market history
available. As important as the secondary market is the public
information about the probability of obtaining the contents of the
boxes, as required by Chinese Law. Because the supply of the boxes is
strictly controlled, the market for these items is lively, with many
items trading for hundreds of dollars, and a few entering the thousands.


These factors combine to allow for a structural estimation of demand,
and risk-tolerance. I intend to combine a demand estimation BLP (1995)
model with Cumulative Prospect Theory to estimate the monetary value
of randomization in the market for weapon skins in Counter-Strike
Global Offensive.

\section{Literature Review}

The literature in demand estimation is primarily focused around the
seminal paper written by \cite*{BLP}. This paper presents a frameowrk
for estimation of a heterogenous consumers in a discrete choice logit
demand framework. This allows for a richer substitution framework, and
an ability for the substitution affects to escape the independence of
irrelevant alternatives result of traditional logit demand.

One such example of estimation is the paper by \cite*{Cereal}. It
estimates the demand in the cereal industry in order to determine the
market power in the industry, and determine if the high product
margins were caused by brand recognition, or by collusive behavior
between the few firms in the industry. This type of counter-factual
estimation is common within the literature, and is tested in many ways
from both the supply and the demand side of the estimation.

I intend to take a different path from what is commonly performed with
these tools, and attempt to use the estimated parameters to compute
what these consumers would have been willing to pay for an item under
some different policy regime (no randomness)

Estimation of these models began with the strategy first suggested by
\cite{BLP} commonly referred to as the Nested Fixed Point Algorithm,
but has recently been superseded by the Mathematical Programming under
Equality Constraints suggested by \cite*{MPEC}. This algorithm
performs extremely well under sparse Hessian and gradients, of which
my method contains many. This will allow for significantly easier
estimation of the demand system.

\cite*{LitReview} presents a broad review of where Prospect theory has
been applied, as well as its problems with its application,
particularly in the choice of a reference point, which appears to be
very significant, but there is little guidance on what to choose
beyond possibly the expected value of the lottery. Applications of the
model, originally proposed by \cite*{Kahn} exist mostly in finance and
insurance. I intend to extend this body to look at the behavior of
non-expert individuals in a market scenario. I believe that this area
has not had many applications, likely because of the rarity of quality
data outside of these fields.



\section{Data}

The data are market transaction history for all items sold on the
\emph{Steam Community Market} for Counter-Strike: Global
Offensive.

Counter-Strike Global Offensive is a first-person shooter game where
one team (terrorists) attempt to plant a bomb and defend it while the
counter-terrorists attempt to defuse the bomb. Each team has specific
guns that they are able to purchase at the start of every round. The
in-game cost, game balance, and meta-game all contribute to the
popularity of each weapon. Players may choose to purchase purely
cosmetic ``skins'' for their weapons which change the appearance of
their weapon when they buy it. These skins are sold in lotteries
called weapon crates which are dropped randomly to players
in-game. The drop rates are unknown, and believed to change
often. Upon receiving a weapon crate, a player may elect to spend
\$2.50 to open it, or sell it on the community market.

These crates display which weapon skins they may contain, and the
probability of obtaining each item within the crate is public
knowledge. That is, the contents of the crate follow a known
distribution, and can therefore be estimated under theories of
risk. The contents of the crate can then be held onto, or sold at
market.

\subsection{Market}


The market that these weapons can be sold at is the \emph{Steam
  Community Market} which is run by Valve, the same company that makes
Counter-Strike: Global Offensive. The market is a continuous time
double-auction. Sellers may place sell orders, and buyers buy orders,
and the market functions by matching the buyers and sellers, always
selling at the seller's price. This is known to converge quickly to a
competitive market, and will be treated as such for this
project. \cite{Efficiency} There are two complications however, there is a 15\% tax
placed on the market by Valve, which is taken from the seller's
earnings. This is complicated by the discrete nature of the selling,
and the tax always rounds up in favor of Valve. That is, an item
selling for \$0.03 would return \$.02 to Valve rather than 15\%. This
will not be a large factor in my model as I am primarily interested in
calculating demand.

For the past 30 days, there is data on hourly median market price as
well as quantity sold. For the remaining time that an item has been at
market, there is data for daily median price and quantity sold. The
data also contain active buy and sell orders at the time of its
mining: (June $7^{th}$ 2018). No history for these buy and sell orders
is available.

\subsection{Characteristics}

Since the model used will be in the characteristic space rather than
the product space, I am especially interested in characteristics of
the different weapons in the game. I shall ignore the characteristics
that will be used to determine the market for the weapon, detailed in
Assumption 1 in the model section. Unique to each weapon is a float
value, between $0$ and $1$, which indicates the wear on the
weapon. Wear does not change with use, and is determined when a weapon
is un-boxed. This float is distributed uniformly, but based on its
value, places the weapon into different brackets for sale. We will
consider all weapons in a particular bracket as homogenous.
\begin{center}
\begin{tabular}{|l|l|}\hline
  Float & Condition\\\hline
  0.00 - 0.07 & Factory New\\
  0.07 - 0.15 & Minimal Wear\\
  0.15 - 0.38 & Field-Tested\\
  0.38 - 0.45 & Well-Worn\\
  0.45 - 1.00 & Battle-Scarred\\\hline
\end{tabular}
\end{center}

Independent of wear, each item
also has a $10\%$ chance of being StatTrak\texttrademark, where the
gun includes a tracker that counts the number of kills a player has
with this weapon. This number is reset on sale, so it can be treated
simply as a binary indicator.

The contents of the crate are divided into several tiers, based on
their rarity from being obtained in a box. These tiers and their
probability of being obtained are given below:

\begin{center}
\begin{tabular}{|c|l|}\hline
Probability & Rarity\\\hline
.0026 & Special (Gold)\\
.0064 & Covert (Red)\\
.032 & Classified (Pink)\\
.1598 & Restricted (Purple)\\
.7992 & Mil-spec (Blue)\\\hline
\end{tabular}
\end{center}

\section{Model}

I intend to estimate a structural model for the demand for the
contents of the boxes, using this, we can determine the distribution
of valuations for a risk-neutral consumer for the boxes, and then
estimate the risk-preference of the individuals that open the
loot-boxes. From there we can calculate the benefit of randomization
compared to selling each item at market.

\subsection{Demand Estimation}

We wish to estimate the demand for this model using a discrete choice
model for demand. This immediately raises the concern that it only
allows for one good to be purchased, and it is common for individuals
to have many weapon skins in the game. To this end, we shall split the
market into several sub-markets and make a heavy identifying
assumption. This assumption will allow for the discrete choice model
to be applicable, and also creates price instruments for estimation.

\begin{assume}
  Items are split into markets defined by the in-game role that all of
  the weapons in this market fulfill.
\end{assume}

These markets are defined by domain knowledge. For example, we treat
the AK-47, the single most popular gun in the game as its own market,
competing only with its own skin and condition variants. However, the
M4A4 and the M4A1-S will be considered as competitors, as will the
CZ75, Tec9, and Five-Seven. Weapons that fill the same role, or the
same weapon slot will be considered in the same market. The assumption
takes the form of claiming that one individuals do not substitute
between roles, and only consider substitution between weapon skins for
the same role. This ensures that consumers only purchase a single item
at a time, as one could never equip multiple skins for the same
role. The full power of this assumption will become clear in the
instruments section.

\subsubsection{BLP}

To estimate the demand for the contents of the boxes, I intend to
implement a standard BLP demand estimation model (1995). This is a discrete
choice demand system. Consider $J$ goods in $T$ markets for $I$
consumers indexed by $j,t,i$ respectively. Assuming quasilinear
utility, the utility for consumer $i$ purchasing good $j$ is:

\begin{equation*}
  u_{ij} = \alpha_i p_j + x_j \beta_i + \xi_j + \epsilon_{ij}
\end{equation*}

Where $p_j$ is the price of good $j$, $x_j$ are the observed
characteristics of good $j$, $\xi_j$ are the characteristics of good
$j$ observed by consumers and producers but not by the econometrician,
$\alpha_i, \beta_i$ are consumer i's individual preference parameters over
these characteristics, and $\epsilon_{ij} \sim T1EV(0)$ is a random shock only
observed by the consumer. This is a standard logit model, but we have
unobserved heterogeneity among consumers.

Consumer $i$ then chooses the good that gives him the highest utility,
the probability that that good is good $j$ is given by:
\begin{equation*}
  \Pr( i \rightarrow j ) = \frac{\exp( \alpha_i p_j + x_j' \beta_i + \xi_j)}{\sum_{k \in
      \mathcal{F}_t} \exp( \alpha_i p_k + x_k' \beta_i + \xi_k)}
\end{equation*}

Each consumer has individual logit demand. If we choose to normalize
the mass of consumers to one, then the market share of good $j$ should
be equal to the expected value of this individual demand, averaged
over the distribution of valuations.

\begin{equation*}
  \pi_j = \exV{ \Pr( i \rightarrow j )}
\end{equation*}

Let us define the observed market shares as:
\begin{equation*}
  \hat{s}_j = \frac{1}{I} \sum_{i = 1}^I \indicate{y_i = j}
\end{equation*}

From the Weak Law of Large Numbers, we believe that $\hat{s}_j \plim
\pi_j$. Define the distribution of $(\alpha_i, \beta_i)$ as $\theta$. By assuming that
this convergence in probability has been reached, we arrive at:

\begin{equation*}
  \hat{s}_j \approx \exV{ \Pr( i \rightarrow j )} = \int \Pr( i \rightarrow j) d\theta \approx \frac{1}{N_s}
  \sum_{i=1}^{N_s} \Pr( i \rightarrow j)
\end{equation*}

Where we approximate the integral of $\int \Pr( i \rightarrow j)d\theta$ through any
numerical integration technique. This expression can then be inverted
to solve for $\xi_j$, which is unobserved.

\subsubsection{Instruments}

In this specification of the model, there are two sets of endogenous
variables. Price is obviously correlated with the unobserved
characteristics of the model, but market share is also endogenous
within the model. We shall require two sets of instruments, one for
price, and one for market share.

Valid price instruments are those that are correlated with supply
shocks, but are not correlated with the demand shocks in the
model. It is worth defining precisely what are the supply and demand
for this model.

The supply for each weapon skin is the set of people who have opened
the loot box that contains that item and have elected to sell
it. Shocks that will affect this are changes in consumer tastes
leading to less people choosing to sell, as well as changes in the
drop rates of the crates, controlling the flow of this item into the
market.

Demand for this good is the individuals who elect to buy the good at
the market rather than attempt to earn it through opening loot
boxes. The shocks that affect these people are entrance and exit to
the market as well as changes in taste. (Needs more here)

A Valid price instrument is something that is correlated with supply
shocks, but not with the demand shocks. For this we will take the
prices of the other contents of the box that are not in the same
market as the good at hand. By Assumption 1, these prices are
exogenous to the unobserved characteristics of the good at hand. They
are however affected by the changes in the drop rate of the loot box
that provides them, since they come from (nearly) the same
supply. This is a form of the Hausman instruments used often in the
literature.

For market share, we intend to use the BLP instruments, which require
that the valuation of one characteristic of a good is not random
across the consumers. When this is satisfied, we may use the sum of
the characteristics of the competitors of the good as instruments for
the market share. If necessary, following \cite{OptimalBLPInstrument}, we
may construct higher order approximations of the optimal instrument
for the market shares using the observed characteristics.

\subsubsection{Estimation}

Once a set of instruments has been computed, estimation of the model
requires using the orthogonality condition of the instruments against
the computed values of $\xi_j$. Our orthogonality condition is:
$\exV{\xi_j z_j} = 0$.

This can be estimated using the generalized method of
moments. Following the method of \cite*{MPEC}, we may estimate
this using Mathematical Programming under Equality Constraints as
follows: 

\begin{align}
  &\min_{\bm{\xi}_{j,t}, \xi_{j,t}} \bm{\xi}_{j,t}' \Omega \bm{\xi}_{j,t}\\
    \text{subject to: } &s_{j,t} = \frac{1}{N_s} \sum_{i=1}^{N_s}
                          \frac{\exp(\alpha_i p_j + x_j'\beta_i + \xi_j)}{\sum_{k\in
                          \mathcal{F}_t} \exp( \alpha_i p_k + x_k'\beta_i +
                          \xi_k)}\\
  &\bm{\xi}_{j,t} = \xi_{j,t} \bm{z}_{j,t}  
\end{align}

This  method allows for the exploitation of sparseness in many
commercial solvers. This is important as assumption 1 has imposed this
level of sparseness on the model in part for computational ease.

\subsection{Risk Preference}

Following the estimation of (1) we may back out the parameters of the
distribution $\theta$. These parameters form the distribution of valuations
for each of the contents of the loot box. The risk-neutral
distribution of valuations can be calculated as this valuation is a
convex combination of the valuations for all the contents. Of interest
is how this distribution varies with the distribution of valuations
observed by those purchasing loot boxes. Since these distributions
have been assumed to be normally distributed, we obtain the
distribution of these valuations, as the sum of normally distributed
random variables is normally distributed as well.

However, it is not immediate that these two distributions are at all
comparable. There must be some structural difference between
individuals who choose to open the boxes, and individuals who choose
to buy the goods at the secondary market, or they would be performing
the same action. Either there is no heterogeneity between individuals,
and the market is in equilibrium, or there is heterogeneity and these
two markets exist to separate the two types.

The simpler of these two explanations is that there is no
heterogeneity among risk for individuals, and that these two markets
are in equilibrium.

\begin{assume}
  Individuals are homogenous in risk-preferences
\end{assume}

In this case, the individuals that purchase the
loot boxes have exactly the same distribution of valuations for the
items as do the individuals that purchase the items at market. We wish
to estimate this risk tolerance by the difference in the price
observed at market and the risk-neutral distribution.


\subsubsection{Cumulative Prospect Theory}

However, these lotteries contain a mixture of small losses and large
gains. Expected utility theory is relatively poor at explanatory power
in this area, and for this reason I choose to estimate the primitives
using Cumulative Prospect Theory.

Cumulative Prospect Theory includes several important concepts not
observe in classical decision making under risk.It incorporates
reference dependence, which implies that people view things in the
context of losses and gains rather than changes to their overall
wealth. This is attractive for computational reasons. It also utilizes
loss aversion, the notion that losses are relatively more costly than
gains. The model also incorporates diminishing sensitivity, the notion
that valuations are concave in losses and convex in gains. The final
concept is probability weighting, the notion that consumers act as if
they were facing different probabilities than what they encounter.

Maintaining with the function forms first suggested by \cite*{Kahn},
I will attempt to find the parameter values that maximize the
likelihood of obtaining the prices observed in the market.

We note that the price of opening a case is two-fold, first the case
must be bought at the market for its price, and then the price of the
key, denoted $p_{key}$ must also be paid to open the case. The time
required to open the case is trivial, and will not be
considered. Since gains are treated differently than losses, let the
parameterization of these gains and losses be defined as:

\begin{equation}
  \label{piecewiseValuation}
  v(x) =
  \begin{cases}
    x^\alpha \quad &x \geq 0\\
    -\lambda(-x)^\alpha \quad &x < 0
  \end{cases}
\end{equation}

In cumulative prospect theory, the cumulative mass (distribution)
function is weighted such that individuals overweight the tail
probabilities. This is especially important in this model, as there are
many high valued rare items, that if this part of the theory is
correct, heavily influence the valuation of the box, despite their
extremely low probability of occurrence.

\begin{equation}
  \label{weightFunction}
  w(P) = \frac{ P^\delta }{( P^\delta + (1-P)^\delta )^{\frac{1}{\delta}}}
\end{equation}

To define the decision weights $\pi_i$, we must first order the prospects
of the lottery in ascending order of gains. the weight $\pi_i$ then is
defined by:

\begin{equation}
  \label{piWeights}
  \pi_i = w \left[ \sum_{j = -m}^i P(x_j) \right] - w \left[ \sum_{j=-m}^{i-1} P(x_j) \right]
\end{equation}

These can all be combined to form the Valuation Transform for the
consumer:

\begin{equation}
\label{Valuation}
F(V_i) = \begin{cases}
\left[w\left( \sum_{j = -m}^i P(x_j) \right) - w\left( \sum_{j=-m}^{i-1} P(x_j) \right)\right](V_i - p_l - p_{key})^\alpha \quad &(V_i - p_l - p_{key}) \geq 0 \\\\
-\lambda \left[  w\left( \sum_{j = -m}^i P(x_j) \right) - w\left( \sum_{j=-m}^{i-1} P(x_j) \right)\right] (p_l + p_{key} - V_i)^\alpha \quad &(V_i - p_l - p_{key}) < 0 \\
\end{cases}
\end{equation}


Under cumulative Prospect theory the distribution of valuations for
a lottery in the population is given by the distribution of:
\begin{equation}
  \sum_{j=-m}^n  F(V_j)  
\end{equation}

% Since the distribution of $\theta$ is normal, Guassian Quadrature can be
% used to evaluate this function relatively cheaply.

% While this distribution may be difficult to understand analytically,
% it can be sampled using MCMC techniques. Is this the path we want to
% go down?

This is then the demand for the loot box as a function of the
cumulative prospect parameters: $\alpha, \delta, \lambda$. It is important to note
that the act of purchasing only implies that an individual has a
valuation higher than the market price. Only for active buy orders can
we take the valuation as non-censored.

These parameters can then be estimated using Censored Maximum
likelihood Estimation. This presents its own concerns as this is a
non-convex optimization problem, however since the dimension of the
problem is so small, an intractable problem such as this may still be
solvable. If this function is too computationally difficult to estimate, it can
be simulated using Monte-Carlo methods, and the likelihood taken from
the kernel-smoothed density function of the simulation.

Once these parameters have been estimated, we can then compute the
monetary benefit of the randomization. This is defined as the
difference between the valuation of the loot box under cumulative
prospect theory, and the risk-neutral valuation of the loot box. This
value integrated over the distribution of consumer types is the total
value gained from the risk preferences of consumers.



\begin{equation}
  \label{eq:ValveProfit}
  \Pi = \int \sum_{j=-m}^n \left[ F(V_j) - V_j \right] d\theta
\end{equation}

%\nocite{OptimalBLPInstrument}
%\nocite{SteamMarket}

\bibliography{biblio}{}

\end{document}
