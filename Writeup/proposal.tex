\documentclass[12pt]{paper}



\usepackage[margin=1in]{geometry}
\usepackage{tikz}
\usepackage{natbib}
\usepackage{amsmath}
\usepackage{bm}
\usepackage{amsthm}
\usepackage{mathtools}
\usepackage{amsfonts}
\usepackage{bbm}
\usepackage{graphicx}

\DeclareMathOperator{\diam}{diam}
\DeclareMathOperator{\interior}{int}
\DeclareMathOperator{\close}{cl}

\newcommand{\met}[1]{d \left ( #1 \right )}
\newcommand{\brak}[1]{ \left [ #1 \right ] }
\newcommand{\cbrak}[1]{ \left \{ #1 \right \}}
\renewcommand{\vec}[1]{ \bm{ #1 }}
\newcommand{\abs}[1]{\left \lvert #1 \right \rvert}
\newcommand{\seq}[1]{{\left \{ #1 \right \}}}
\newcommand{\conj}[1]{ \overline{ #1 } }
%\newcommand{\close}[1]{ \bar{ #1 } }
\newcommand{\set}[1]{\left \{ #1 \right \}}
\newcommand{\Lim}{\lim\limits}
\newcommand{\compose}{\circ}
\newcommand{\inv}[1]{{#1}^{-1}}
\newcommand{\compl}[1]{{#1}^{c}}



\newcommand{\setR}{ \mathbb{R} }
\newcommand{\setQ}{ \mathbb{Q} }
\newcommand{\setZ}{ \mathbb{Z} }
\newcommand{\setN}{ \mathbb{N} }

\newcommand{\plim}{ \overset{p}{\to} }
\newcommand{\mean}[2][N]{ \overline{ #2 }_{#1}}
\newcommand{\exV}[1]{\mathbb{E} \left [ #1 \right ]}
\newcommand{\Vari}[1]{\mathbb{V} \left ( #1 \right )}

\newcommand{\est}[2][n]{ \widehat{ #2 }_{#1}}
\newcommand{\altest}[2][n]{ \tilde{ #2 }_{#1}}

\newcommand{\indicate}[1]{ \mathbbm{1}_{\{#1\}}}
\newcommand{\convDist}{ \overset{d}{\to}}
\newcommand{\unif}{\emph{U}}
\newcommand{\normal}{\mathcal{N}}
\newcommand{\eye}{\mathbbm{I}}

\newcommand{\bigO}{\mathcal{O}}
\newcommand{\Lagrange}{\mathcal{L}}

\newcommand{\deriv}[2]{\frac{ \partial #1}{ \partial #2}}

\DeclarePairedDelimiter{\ceil}{\lceil}{\rceil}
\DeclarePairedDelimiter{\floor}{\lfloor}{\rfloor}
\DeclarePairedDelimiter{\norm}{\lVert}{\rVert}

\newtheorem{assume}{Assumption}

\bibliographystyle{chicago}

\title{The benefits of Randomization Mechanisms in Counter-Strike:
  Global Offensive}
\author{Timothy Schwieg}

\begin{document}

\maketitle

\section{Research Question}


In the world of video games, a market has appeared for in-game
purchases. These cosmetic items affect the aesthetics of a player, but
often do not influence the game-play and are sold by the
designer. Recently the method of the sales has moved away from the
traditional market approach of individual prices for each item, and
towards the ``loot box'' approach. These items are sold in randomized
lotteries, often given away, with a cost of opening.

Traditional economic research on randomization informs us that for the
risk-neutral customer, there is no benefit to randomization, as the
consumer is indifferent. So for this mechanism to be so far-reaching
into the market, there must be a risk-loving nature to the
consumers. This begs the question of how much money are these
companies gaining by exploiting the risk-loving nature of the
consumers.

Counter-Strike global offensive presents an interesting case study for
these types of markets, as there is a secondary market where
individuals can buy and sell these loot boxes, as well as their
contents. This was one of the first games to introduce the concept of
the randomized ``loot box'' so there is a long market history
available. As important as the secondary market is the public
information about the probability of obtaining the contents of the
boxes, as required by Chinese Law. Because the supply of the boxes is
strictly controlled, the market for these items is lively, with many
items trading for hundreds of dollars, and a few entering the thousands.


These factors combine to allow for a structural estimation of demand,
and risk-tolerance. I intend to combine a discrete choice demand
estimation model with Cumulative Prospect Theory to estimate the
monetary value of randomization in the market for weapon skins in
Counter-Strike Global Offensive.

\section{Literature Review}

The literature in demand estimation is primarily focused around the
seminal paper written by \cite*{BLP}. This paper presents a framework
for estimation of a heterogenous consumers in a discrete choice logit
demand framework. This allows for a richer substitution framework, and
an ability for the substitution affects to escape the independence of
irrelevant alternatives result of traditional logit demand.

One such example of estimation is the paper by \cite*{Cereal}. It
estimates the demand in the cereal industry in order to determine the
market power in the industry, and determine if the high product
margins were caused by brand recognition, or by collusive behavior
between the few firms in the industry. This type of counter-factual
estimation is common within the literature, and is tested in many ways
from both the supply and the demand side of the estimation.

I intend to take a different path from what is commonly performed with
these tools, and attempt to use the estimated parameters to compute
what these consumers would have been willing to pay for an item under
some different policy regime (no randomness)

Estimation of these models began with the strategy first suggested by
\cite{BLP} commonly referred to as the Nested Fixed Point Algorithm,
but has recently been superseded by the Mathematical Programming under
Equality Constraints suggested by \cite*{MPEC}. This algorithm
performs extremely well under sparse Hessian and gradients, of which
my method contains many. This will allow for significantly easier
estimation of the demand system.

\cite*{LitReview} presents a broad review of where Prospect theory has
been applied, as well as its problems with its application,
particularly in the choice of a reference point, which appears to be
very significant, but there is little guidance on what to choose
beyond possibly the expected value of the lottery. Applications of the
model, originally proposed by \cite*{Kahn} exist mostly in finance and
insurance. I intend to extend this body to look at the behavior of
non-expert individuals in a market scenario. I believe that this area
has not had many applications, likely because of the rarity of quality
data outside of these fields.

The literature on Cumulative Prospect theory, the main structure of
this model is primarily focused on experimental data. 


\section{Data}

The data are market transaction history for all items sold on the
\emph{Steam Community Market} for Counter-Strike: Global
Offensive.

Counter-Strike Global Offensive is a first-person shooter game where
one team (terrorists) attempt to plant a bomb and defend it while the
counter-terrorists attempt to defuse the bomb. Each team has specific
guns that they are able to purchase at the start of every round. The
in-game cost, game balance, and meta-game all contribute to the
popularity of each weapon. Players may choose to purchase purely
cosmetic ``skins'' for their weapons which change the appearance of
their weapon when they buy it. These skins are sold in lotteries
called weapon crates which are dropped randomly to players
in-game. The drop rates are unknown, and believed to change
often. Upon receiving a weapon crate, a player may elect to spend
\$2.50 to open it, or sell it on the community market.

These crates display which weapon skins they may contain, and the
probability of obtaining each item within the crate is public
knowledge. That is, the contents of the crate follow a known
distribution, and can therefore be estimated under theories of
risk. The contents of the crate can then be held onto, or sold at
market.

\subsection{Market}


The market that these weapons can be sold at is the \emph{Steam
  Community Market} which is run by Valve, the same company that makes
Counter-Strike: Global Offensive. The market is a continuous time
double-auction. Sellers may place sell orders, and buyers buy orders,
and the market functions by matching the buyers and sellers, always
selling at the seller's price. This is known to converge quickly to a
competitive market, and will be treated as such for this
project. \cite{Efficiency} There are two complications however, there is a 15\% tax
placed on the market by Valve, which is taken from the seller's
earnings. This is complicated by the discrete nature of the selling,
and the tax always rounds up in favor of Valve. That is, an item
selling for \$0.03 would return \$.02 to Valve rather than 15\%. This
will not be a large factor in my model as I am primarily interested in
calculating demand.

For the past 30 days, there is data on hourly median market price as
well as quantity sold. For the remaining time that an item has been at
market, there is data for daily median price and quantity sold. The
data also contain active buy and sell orders at the time of its
mining: (June $7^{th}$ 2018). No history for these buy and sell orders
is available.

\subsection{Characteristics}

Since the model used will be in the characteristic space rather than
the product space, I am especially interested in characteristics of
the different weapons in the game. I shall ignore the characteristics
that will be used to determine the market for the weapon, detailed in
Assumption 1 in the model section. Unique to each weapon is a float
value, between $0$ and $1$, which indicates the wear on the
weapon. Wear does not change with use, and is determined when a weapon
is un-boxed. This float is distributed uniformly, but based on its
value, places the weapon into different brackets for sale. We will
consider all weapons in a particular bracket as homogenous.
\begin{center}
\begin{tabular}{|l|l|}\hline
  Float & Condition\\\hline
  0.00 - 0.07 & Factory New\\
  0.07 - 0.15 & Minimal Wear\\
  0.15 - 0.38 & Field-Tested\\
  0.38 - 0.45 & Well-Worn\\
  0.45 - 1.00 & Battle-Scarred\\\hline
\end{tabular}
\end{center}

Independent of wear, each item
also has a $10\%$ chance of being StatTrak\texttrademark, where the
gun includes a tracker that counts the number of kills a player has
with this weapon. This number is reset on sale, so it can be treated
simply as a binary indicator.

The contents of the crate are divided into several tiers, based on
their rarity from being obtained in a box. These tiers and their
probability of being obtained are given below:

\begin{center}
\begin{tabular}{|c|l|}\hline
Probability & Rarity\\\hline
.0026 & Special (Gold)\\
.0064 & Covert (Red)\\
.032 & Classified (Pink)\\
.1598 & Restricted (Purple)\\
.7992 & Mil-spec (Blue)\\\hline
\end{tabular}
\end{center}

\section{Model}

\subsection{Discrete Choice Demand}



Let us believe that individuals have a valuation for loot boxes as
suggested by Cumulative Prospect Theory. Following a discrete choice
framework for demand estimation, I assume that the utility of a
consumer $i$ for loot box $j$ in time $t$ is given by:
\begin{equation*}
  u_{ijt} = V( x_{jt}, p_{jt}; \theta ) + \xi_{jt} + \epsilon_{ij} \quad \epsilon_{ij} \sim Gumbel
\end{equation*}

Where $V$ is the valuation for the loot box, $p_{jt}$ is the price,
$x_{jt}$ are the covariates, $\xi_{jt}$ is some demand shock common to
all consumers (this can be rationalized as unobserved benefits), 
$\epsilon_{ij}$ is a type-1 extreme value shock unique to the consumer and
good, and $\theta$ is the vector of parameters for the valuation function

The demand for this good then is given by the probability that it has
the maximum utility. This can be computed using the properties of the
Type-1 extreme value distribution. The maximum follows a logistic
distribution, and the probability is given by:

\begin{equation*}
  \Pr( i \rightarrow j ) = \frac{\exp( V(x_{jt},p_{jt} ; \theta) + \xi_{jt})}{ \sum_{k \in \mathcal{F}}
    \exp(V(x_{jt},p_{jt}; \theta) + \xi_{kt})}
\end{equation*}

In this sense, demand is non-random, and the Econometrician observes
the price of the box, the covariates of the box, as well as the
equilibrium quantity $q_{jt}$. All facets here observed, save the fact
that the price and quantity are equilibrium prices and quantity rather
than various points along the same demand curve.

Following the structure of Berry (1994) we consider an outside option
that has some market share. The outside option is simply not partaking
in any of the lotteries, and thus the valuation of this is $0$.
However there is still some unobserved demand $\xi_{0t}$. Inversion to
solve for this parameter is simple, as
$\Pr( i \rightarrow 0) = \frac{\xi_0}{ \sum_{k \in \mathcal{F}} \exp(V(x_{jt},p_{jt}; \theta) +
  \xi_k)}$.  Dividing each demand equation by the outside option and
taking logs yields us:

\begin{equation*}
  \log s_{jt} - \log s_{0t} = V(x_{jt}, p_{jt}; \theta) + \xi_{jt}
\end{equation*}

Since $\xi_{jt}$ is unobserved by the econometrician, it takes the form
of the unobserved error in the demand estimation procedure. However,
it is sometimes endogenous to price as price is formed by the
intersection of both supply and demand shocks. We need valid
instruments for the estimation of this demand.

\subsection{Instruments}

Endogeneity occurs in this model via the simultaneity of supply and
demand. Valid instruments for the price therefore must be supply
shifters that do not affect the demand. Supply can be thought of as
the players who have received a loot box randomly and wish to sell
it. To simplify the dynamics of the problem, upon receiving the item
individuals plan to sell it or not, so the supply of these loot boxes
is heavily dependent on active players in that day and the previous
day. 

It is assumed that demand is a function of the long-run average number
of players, or the amount of ``active players'' over the period of the
month. This number is different from the daily players that play each
day, as relatively few people are able to play each day for many
reasons. However, loot boxes are given randomly to each player who
plays in a day. We wish to use this fact to construct instruments for
the demand. Suppose the true number of active players is $N$. Then daily
players is $N + \epsilon_t$, i.e. some shock that determines daily
player-base. We wish to use this shock $\epsilon_t$ as a cost-shifter that
does not affect demand. If we estimate N by the average of all players
over a significant time period, we can instrument demand using the
daily deviations from this average. We instrument for price with the
deviations of the current day of sale as well as the previous days.


Supply can be thought of as upward sloping with an active price floor
at a price of $.03$ which is often binding. In the set of transactions
where the price floor is binding, there is no concern of simultaneity,
and therefore price is exogenously determined by the existence of the
price floor. Since we use multiple instruments for price in the
endogenous case, the remaining instruments will be made zero for the
exogenous price case. 

Demand can then be estimated off of the condition that:
\begin{equation*}
  \exV{Z_t (\xi_{tj})} = 0 \quad \quad \quad \exV{ p_{jt} \xi_{jt} } = 0 \quad \text{When } p = \$.03
\end{equation*}


Also present in the data is active buy orders, these are orders
that there is not yet supply to fulfill. However it is a dominant
strategy for place your valuation as the bid. Therefore there is no
concerns about shading, and we may treat these orders as true
valuations. In the case of these estimates, we should find that demand
shock is equal to zero, and uncorrelated with the valuation, or that
$\exV{p_{jt}\xi_{jt}} = 0$. I note that the same choice and discrete
choice framework holds for those that post unfulfilled buy orders, so
there is no different model for valuations under the unfulfilled order
framework. 

This gives us price moments for each of the possible cases of the
data. The rest of the covariates are the probabilities of obtaining
each of the items, which are obviously exogenous and the last known
prices of the contents of the loot boxes. We shall takes these prices
as exogenous as they were determined by the supply and demand of the
item in previous time periods.

We may combine all of these into a vector $x_{jt}$ along with a
constant term and our condition becomes one of
$\exV{x_{jt}\xi_{jt}} = 0$. This provides us with $2k + 2$ moments per
data point, when there are $k$ contents, and there are $3$
parameters of interest to estimate. We are extremely over-identified,
and may be able to consider more complicated functional forms given
additional time.


\subsection{Estimation}

\begin{align*}
  \xi_{jt} = \log s_{jt} - \log s_{0t} - V( x_{jt}, p_{jt}; \theta)
\end{align*}


Consider a matrix of exogenous variables $Z_j$ defined as above, we wish
to estimate the parameters of $V$ based on the condition that this
matrix is orthogonal to $\xi$. This could be accomplished using either
Nonlinear Least-Squares or Generalized Method of Moments, I shall
employ the latter.

All of the orthogonality conditions combine to:
\begin{equation*}
  \exV{Z_{jt}'\xi_{jt}} = 0
\end{equation*}

The estimation procedure can be written as:

\begin{align}
  &\min_{\bm{\xi}_{j,t}, \xi_{j,t}} \sum_{j,t}\bm{\xi}_{j,t}' \Omega \bm{\xi}_{j,t}\\
  \text{subject to: } &\xi_{j,t} = \log s_{jt} - \log s_{0t} - V( x_{jt}, p_{jt}; \theta)\\
  &\bm{\xi}_{j,t} = \xi_{j,t} \bm{Z}_{j,t}  
\end{align}



\subsection{Cumulative Prospect Theory}


We now examine the structure of the Valuation function $V( x_j,
p_j; \theta)$. Denote the probabilities of each of the contents of the
lotteries by $\pi_i$ and their values as $x_i$. We now re-index these by
a permutation $s_i$ such that $x_{s_1} < x_{s_2} < ...$ and so on. The
cumulative probability $\Pi_i$ can be written as $\sum_{i=1}^K
\pi_{s_i}$. From these objects we can construct the value function for
the lottery.

Cumulative Prospect Theory includes several important concepts not
observe in classical decision making under risk. It incorporates
reference dependence, which implies that people view things in the
context of losses and gains rather than changes to their overall
wealth. This is attractive for computational reasons. It also utilizes
loss aversion, the notion that losses are relatively more costly than
gains. The model also incorporates diminishing sensitivity, the notion
that valuations are concave in losses and convex in gains. The final
concept is probability weighting, the notion that consumers act as if
they were facing different probabilities than what they encounter.

We note that the price of opening a case is two-fold, first the case
must be bought at the market for its price, and then the price of the
key, denoted $p_{key}$ must also be paid to open the case. The time
required to open the case is trivial, and will not be
considered. Since gains are treated differently than losses, let the
parameterization of these gains and losses be defined as:

\begin{equation*}
  v(x) =
  \begin{cases}
    x^\alpha \quad &x \geq 0\\
    -\lambda(-x)^\alpha \quad &x < 0
  \end{cases}
\end{equation*}

In this sense, $\alpha$ captures the risk-loving or risk-averse nature of
the consumer, while $\lambda$ captures their level of loss-aversion.


In cumulative prospect theory, the cumulative mass (distribution)
function is weighted such that individuals overweight the tail
probabilities. This is especially important in this model, as there are
many high valued rare items. If this is the case, a severely distorted
distribution could lead to individuals systematically overvaluing
lotteries despite even being risk-averse. Colloquially, it is believed
that this concept is what allows for this type of randomization to
flourish in a market that is primarily populated by young people.

\begin{equation*}
  w(P) = \frac{ P^\delta }{( P^\delta + (1-P)^\delta )^{\frac{1}{\delta}}}
\end{equation*}

For some item that is within the contents of the box, a consumer's
valuation is transformed by the function:
\begin{equation*}
F(x_i) = \brak{w( \Pi_{s_i}) - w(\Pi_{s_i - 1}) } v( x_i - p_j - 2.50)
\end{equation*}

The valuation for the lottery can then be written as the sum of this
transformation for all of the contents of the lottery:

\begin{equation*}
  V(x,\pi,p_j) = \sum_{i=1}^K F( x_i)
\end{equation*}


\section{Counterfactuals}

Of interest is how much better this market structure is performing
compared to a monopoly pricing schedule. This would require estimating
the demand for the contents of the lottery, and then computing the
optimal monopolist price for each good. 



% I intend to estimate a structural model for the demand for the
% contents of the boxes, using this, we can determine the distribution
% of valuations for a risk-neutral consumer for the boxes, and then
% estimate the risk-preference of the individuals that open the
% loot-boxes. From there we can calculate the benefit of randomization
% compared to selling each item at market.

 \subsection{Demand Estimation}

We wish to estimate the demand for this model using a discrete choice
model for demand. This immediately raises the concern that it only
allows for one good to be purchased, and it is common for individuals
to have many weapon skins in the game. To this end, we shall split the
market into several sub-markets and make a heavy identifying
assumption. This assumption will allow for the discrete choice model
to be applicable, and also creates price instruments for estimation.

\begin{assume}
  Items are split into markets defined by the in-game role that all of
  the weapons in this market fulfill.
\end{assume}

These markets are defined by domain knowledge. For example, we treat
the AK-47, the single most popular gun in the game as its own market,
competing only with its own skin and condition variants. However, the
M4A4 and the M4A1-S will be considered as competitors, as will the
CZ75, Tec9, and Five-Seven. Weapons that fill the same role, or the
same weapon slot will be considered in the same market. The assumption
takes the form of claiming that one individuals do not substitute
between roles, and only consider substitution between weapon skins for
the same role. This ensures that consumers only purchase a single item
at a time, as one could never equip multiple skins for the same
role. The full power of this assumption will become clear in the
instruments section.

\subsubsection{BLP}

To estimate the demand for the contents of the boxes, I intend to
implement a standard BLP demand estimation model (1995). This is a discrete
choice demand system. Consider $J$ goods in $T$ markets for $I$
consumers indexed by $j,t,i$ respectively. Assuming quasilinear
utility, the utility for consumer $i$ purchasing good $j$ is:

\begin{equation*}
  u_{ij} = \alpha_i p_j + x_j \beta_i + \xi_j + \epsilon_{ij}
\end{equation*}

Where $p_j$ is the price of good $j$, $x_j$ are the observed
characteristics of good $j$, $\xi_j$ are the characteristics of good
$j$ observed by consumers and producers but not by the econometrician,
$\alpha_i, \beta_i$ are consumer i's individual preference parameters over
these characteristics, and $\epsilon_{ij} \sim T1EV(0)$ is a random shock only
observed by the consumer. This is a standard logit model, but we have
unobserved heterogeneity among consumers.

Consumer $i$ then chooses the good that gives him the highest utility,
the probability that that good is good $j$ is given by:
\begin{equation*}
  \Pr( i \rightarrow j ) = \frac{\exp( \alpha_i p_j + x_j' \beta_i + \xi_j)}{\sum_{k \in
      \mathcal{F}_t} \exp( \alpha_i p_k + x_k' \beta_i + \xi_k)}
\end{equation*}

Each consumer has individual logit demand. If we choose to normalize
the mass of consumers to one, then the market share of good $j$ should
be equal to the expected value of this individual demand, averaged
over the distribution of valuations.

\begin{equation*}
  \pi_j = \exV{ \Pr( i \rightarrow j )}
\end{equation*}

Let us define the observed market shares as:
\begin{equation*}
  \hat{s}_j = \frac{1}{I} \sum_{i = 1}^I \indicate{y_i = j}
\end{equation*}

From the Weak Law of Large Numbers, we believe that $\hat{s}_j \plim
\pi_j$. Define the distribution of $(\alpha_i, \beta_i)$ as $\theta$. By assuming that
this convergence in probability has been reached, we arrive at:

\begin{equation*}
  \hat{s}_j \approx \exV{ \Pr( i \rightarrow j )} = \int \Pr( i \rightarrow j) d\theta \approx \frac{1}{N_s}
  \sum_{i=1}^{N_s} \Pr( i \rightarrow j)
\end{equation*}

Where we approximate the integral of $\int \Pr( i \rightarrow j)d\theta$ through any
numerical integration technique. This expression can then be inverted
to solve for $\xi_j$, which is unobserved.

\subsubsection{Instruments}

In this specification of the model, there are two sets of endogenous
variables. Price is obviously correlated with the unobserved
characteristics of the model, but market share is also endogenous
within the model. We shall require two sets of instruments, one for
price, and one for market share.

Valid price instruments are those that are correlated with supply
shocks, but are not correlated with the demand shocks in the
model. It is worth defining precisely what are the supply and demand
for this model.

The supply for each weapon skin is the set of people who have opened
the loot box that contains that item and have elected to sell
it. Shocks that will affect this are changes in consumer tastes
leading to less people choosing to sell, as well as changes in the
drop rates of the crates, controlling the flow of this item into the
market.

Demand for this good is the individuals who elect to buy the good at
the market rather than attempt to earn it through opening loot
boxes. The shocks that affect these people are entrance and exit to
the market as well as changes in taste. (Needs more here)

A Valid price instrument is something that is correlated with supply
shocks, but not with the demand shocks. For this we will take the
prices of the other contents of the box that are not in the same
market as the good at hand. By Assumption 1, these prices are
exogenous to the unobserved characteristics of the good at hand. They
are however affected by the changes in the drop rate of the loot box
that provides them, since they come from (nearly) the same
supply. This is a form of the Hausman instruments used often in the
literature.

For market share, we intend to use the BLP instruments, which require
that the valuation of one characteristic of a good is not random
across the consumers. When this is satisfied, we may use the sum of
the characteristics of the competitors of the good as instruments for
the market share. If necessary, following \cite{OptimalBLPInstrument}, we
may construct higher order approximations of the optimal instrument
for the market shares using the observed characteristics.

\subsubsection{Estimation}

Once a set of instruments has been computed, estimation of the model
requires using the orthogonality condition of the instruments against
the computed values of $\xi_j$. Our orthogonality condition is:
$\exV{\xi_j z_j} = 0$.

This can be estimated using the generalized method of
moments. Following the method of \cite*{MPEC}, we may estimate
this using Mathematical Programming under Equality Constraints as
follows: 

\begin{align}
  &\min_{\bm{\xi}_{j,t}, \xi_{j,t}} \bm{\xi}_{j,t}' \Omega \bm{\xi}_{j,t}\\
    \text{subject to: } &s_{j,t} = \frac{1}{N_s} \sum_{i=1}^{N_s}
                          \frac{\exp(\alpha_i p_j + x_j'\beta_i + \xi_j)}{\sum_{k\in
                          \mathcal{F}_t} \exp( \alpha_i p_k + x_k'\beta_i +
                          \xi_k)}\\
  &\bm{\xi}_{j,t} = \xi_{j,t} \bm{z}_{j,t}  
\end{align}

This  method allows for the exploitation of sparseness in many
commercial solvers. This is important as assumption 1 has imposed this
level of sparseness on the model in part for computational ease.

From these estimates, a monopoly pricing schedule that sets price
where marginal revenue equals marginal cost for all goods can be
computed, and its revenue compared to the revenue generated under the
randomization scheme.


%\nocite{OptimalBLPInstrument}
%\nocite{SteamMarket}

\bibliography{biblio}{}

\end{document}
