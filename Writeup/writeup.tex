\documentclass[12pt]{paper}

\usepackage{Schwieg}

\begin{document}
Let us believe that individuals have a valuation for loot boxes as
suggested by Cumulative Prospect Theory. There is also an additive
error shock that is distributed type-1 extreme value.

Therefore we can say that the utility of a consumer $i$ for loot box
$j$ in time $t$ is given by:
\begin{equation*}
  u_{ijt} = V( x_{jt}, p_{jt} ) + \xi_{jt} + \epsilon_{ij} \quad \epsilon_{ij} \sim Gumbel
\end{equation*}

Where $V$ is the valuation for the loot box, $p_{jt}$ is the price,
$x_{jt}$ are the covariates, $\xi_{jt}$ is some demand shock common to
all consumers (this can be rationalized as unobserved benefits), and
$\epsilon_{ij}$ is a type-1 extreme value shock unique to the consumer and
good.

The demand for this good then is given by the probability that it has
the maximum utility. This can be computed using the properties of the
Type-1 extreme value distribution. The maximum follows a logistic
distribution, and the probability is given by:

\begin{equation*}
  \Pr( i \rightarrow j ) = \frac{\exp( V(x_{jt},p_{jt}) + \xi_{jt}}{ \sum_{k \in \mathcal{F}}
    \exp(V(x_{jt},p_{jt}) + \xi_{kt})}
\end{equation*}

In this sense, demand is non-random, and the Econometrician observes
the price of the box, the covariates of the box, as well as the
equilibrium quantity $q_{jt}$. All facets here observed, save the fact
that the price and quantity are equilibrium prices and quantity rather
than various points along the same demand curve.

Following the structure of Berry (1994) we consider an outside option
that has some market share. The outside option is simply not partaking
in any of the lotteries, and thus the valuation of this is $0$.
However there is still some unobserved demand $\xi_{0t}$. Inversion to
solve for this parameter is simple, as
$\Pr( i \rightarrow 0) = \frac{\xi_0}{ \sum_{k \in \mathcal{F}} \exp(V(x_{jt},p_{jt}) +
  \xi_k)}$.  Dividing each demand equation by the outside option and
taking logs yields us:

\begin{equation*}
  \log s_{jt} - \log s_{0t} = V(x_{jt}, p_{jt}) + \xi_{jt}
\end{equation*}

Since $\xi_{jt}$ is unobserved by the econometrician, it takes the form
of the unobserved error in the demand estimation procedure. However,
it is sometimes endogenous to price as price is formed by the
intersection of both supply and demand shocks. We need valid
instruments for the estimation of this demand. 

Supply can be thought of as upward sloping with an active price floor
at a price of $.03$ which is often binding. However, the supply
changes with the active player base in the day of sale, and the day
before the sale. However in the section where the price floor is
binding, supply is constant and there is no simultaneity
concerns. Price is therefore exogenous in this case, as it is
determined by the price floor. 

Demand can then be estimated off of the condition that:
\begin{equation*}
  \exV{Z_t (\xi_{tj})} = 0 \quad \exV{ p_{jt} \xi_{jt} } = 0 \text{When } p = \$.03
\end{equation*}

We believe that demand is a function of the long-run average number
of players, or the amount of ``active players'' over the period of the
month. This Number is different from the daily players that play each
day, as relatively few people are able to play each day for many
reasons. However, since we have that these loot boxes are given
randomly to each player who plays in a day, We may instrument the
demand by using the difference in the daily player-base from the
long-run average player-base as a cost shifter that is exogenous to
demand. We use today's deviation as well as yesterdays deviation as
instruments. 

Also present in the data is active buy orders, these are orders place
that there is not yet supply to fulfill. However it is a dominant
strategy for place your valuation as the bid. Therefore there is no
concerns about shading, and we may treat these orders as true
valuations. In the case of these estimates, we should find that demand
shock is equal to zero, and uncorrelated with the valuation, or that
$\exV{p_{jt}\xi_{jt}} = 0$

This gives us price moments for each of the possible cases of the
data. The rest of the covariates are the probabilities of obtaining
each of the items, which are obviously exogenous and the last known
prices of the contents of the loot boxes. We shall takes these prices
as exogenous as they were determined by the supply and demand of the
item in previous time periods.

We may combine all of these into a vector $x_{jt}$ along with a
constant term and our condition becomes one of
$\exV{x_{jt}\xi_{jt}} = 0$. This provides us with $2k + 2$ moments per
data point, when there are $k$ contents, and there are 3 parameters of
interest to estimate.

We now examine the structure of the Valuation function $V( x_j,
p_j)$.





\end{document}
