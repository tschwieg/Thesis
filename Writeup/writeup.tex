\documentclass[12pt]{paper}

\usepackage{Schwieg}

\begin{document}
Let us believe that individuals have a valuation for loot boxes as
suggested by Cumulative Prospect Theory. There is also an additive
error shock that is distributed type-1 extreme value.

Therefore we can say that the utility of a consumer $i$ for loot box
$j$ is given by:
\begin{equation*}
  u_{ijt} = V( x_{jt}, p_{jt} ) + \epsilon_{ij} \quad \epsilon_{ij} \sim Gumbel
\end{equation*}

The demand for this good then is given by the probability that it has
the maximum utility. This can be computed using the properties of the
Type-1 extreme value distribution. The maximum follows a logistic
distribution, and the probability is given by:

\begin{equation*}
  \Pr( i \rightarrow j ) = \frac{\exp( V(x_{jt},p_{jt})}{ \sum_{k \in \mathcal{F}}
    \exp(V(x_{jt},p_{jt}))}
\end{equation*}

In this sense, demand is non-random, and the Econometrician observes
the price of the box, the covariates of the box, as well as the
equilibrium quantity $q_{jt}$. All facets here observed, save the fact
that the price and quantity are equilibrium prices and quantity rather
than various points along the same demand curve.

Assume that there exists some outside output, and if we take the log
of the shares and subtract the outside output we arrive at:

\begin{equation*}
  \log s_{jt} - \log s_{0t} = V(x_{jt}, p_{jt})
\end{equation*}

Let the observed equilibrium shares be given by the true demand plus
some unobserved zero mean error $U_D$ that is exogenous to the valuations
$V(x_{jt}, p_{jt})$. We therefore have $\exV{V(x_{jt},p_{jt})U_D} = 0$

The supply of these lotteries is players who are playing the game and
randomly receive the item as a reward for playing. Let us believe that
there is a group of players that simply do not open these loot boxes
upon receiving them. These players then immediately sell their
lotteries on the steam community market. This essentially means that
there is a perfectly inelastic supply of these boxes each
day. However, to complicate matters there is a binding minimum
exchange price of $\$0.03$. At this price, we treat supply as
perfectly elastic, indicating that the supply curve is a corner.

The supply curve can then be given as $q_{jt}^S = \xi_{j} N_t$ when
$p_j > .03$ and $p = .03$ otherwise. There is no randomness in the
quantity supplied in the case where the price floor is binding. So
endogeneity in prices only occurs when the price is non-binding. Price
is also not endogenous for the active buy orders. It is only in the
case when the price is determined by the intersection of supply and
demand that the price is endogenous. We divide the supply by the
players that are searching for weapons denoted $N^{*}$. This is the
total number of players that demand weapons, and $\frac{q_{jt}^S }{N^{*}}
= s_{jt}$. Therefore we can write supply as $s_{jt} = \xi_j^{\prime} N_t$
where both terms have been dividing by $N^{*}$. Taking logs gives us:

\begin{equation*}
  \log s_{jt} = \log \xi_j + \log N_t - \log N^{*}
\end{equation*}

We can note immediately that since neither $\xi_j$ or $N^{*}$ are
observed, we cannot identify these parameters, however since $N^{*}$
is fixed across goods, we could identify differences in the drop
rates: $\xi_j - \xi_k$.

We can formulate this problem as a GMM estimation procedure using the
notion of Mathematical Programming under Equality Constraints:

\begin{align*}
  \min \quad & e_{jt} W e_{jt}\\
  \text{subject to: } e_{jt} &= V(x_{jt}, p_{jt}) - \log s_{jt} + \log
  s_{0t}\\
  \log s_{jt} &= \log \delta_j + \log N_t \text{ when } p_{jt} > .03
\end{align*}



\end{document}
